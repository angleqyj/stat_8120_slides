\documentclass{article}

\usepackage{amsmath}
\usepackage{amsfonts}
\usepackage{url}
\usepackage{bbm} %for bold indicator func

\title{A More Detailed Proof of Prop. 9.3.7}

\begin{document}
\maketitle


\section{Statement}

Recall $\tilde{\mathsf{C}} = \left\{ f \in L^1(\mathsf{X}, \mu) : x \mapsto L(x, |f|) \in \mathsf{C}  \right\}$ and assume 

1. Assumption 9.3.1: $0 < L(x,\mathsf{X}) < \infty$.

2. Assumption 9.3.2: $\{(\xi^{N,i},1)\}_{1 \le i \le M_N}$ are consistent for $(\nu, \mathsf{C})$. $L(x,\mathsf{X}) \in \mathsf{C}$.

3. Assumption 9.3.3: $\forall x \in \mathsf{X}$, $L(x, \cdot) \ll R(x, \cdot)$, and there exists a strictly positive RN derivative: $\frac{dL(x,\cdot )}{dR(x, \cdot)}$.

4. Assumption 9.3.6: the weighted sample $\{(\xi^{N,i}, 1)\}_{1 \le i \le M_N}$ is asymptotically normal for $(\nu, \mathsf{A}, \sigma, M_N^{1/2})$, where $\mathsf{A}$ is proper, and $\sigma$ is some nonnegative function on $\mathsf{A}$.

Then $\tilde{\mathsf{A}}$ is proper, and $\{ (\tilde{\xi}^{N,j}, \tilde{\omega}^{N,j} )\}_{1 \le j \le M_N}$ is asymptotically normal for $(\mu, \tilde{\mathsf{A}}, \tilde{\sigma}, M_N^{1/2})$.





\section{Overall Strategy}

We already showed that $\tilde{\mathsf{A}}$ is proper in class. Now we want to show that asymptotic normality part. So we're going to use Slutsky's to show that
\begin{align*}
\sqrt{M_N} \frac{1}{\tilde{M}_N} \sum_{i=1}^{\tilde{M}_N} \tilde{\omega}^{N,i}\left[ f(\tilde{\xi}^{N,i}) - \nu L(f) \right]
\bigg/ \frac{1}{\tilde{M}_N} \sum_{j=1}^{\tilde{M}_N} \tilde{\omega}^{N,j}
\end{align*}
converges to a normal distribution. The denominator is consistent, the numerator is asymptotically normal.

Unlike the book, I am not assuming that, WLOG, the mean is $0$.

The denominator is the easiest part. $L(x, |1|) = L(x,\mathsf{X}) \in \mathsf{C}$ by assumption 9.3.2, so this means $1 \in \tilde{\mathsf{C}}$. Theorem 9.3.5 from last class gives us that the denominator converges in probability to $\nu L(\mathsf{X})$.

Regarding the numerator, write it as 
\begin{align*}
&\sqrt{M_N}\left\{  \frac{1}{\tilde{M}_N} \sum_{i=1}^{\tilde{M}_N} \tilde{\omega}^{N,i} f(\tilde{\xi}^{N,i}) -  \frac{1}{\tilde{M}_N} \sum_{i=1}^{\tilde{M}_N} E[ \tilde{\omega}^{N,i} f(\tilde{\xi}^{N,i}) \mid \mathcal{F}^N] \right\} \\
& \hspace{10mm} + \sqrt{M_N} \frac{1}{\tilde{M}_N} \sum_{i=1}^{\tilde{M}_N} \left\{ E[ \tilde{\omega}^{N,i} f(\tilde{\xi}^{N,i}) \mid \mathcal{F}^N] - \nu L(f) \right\}\\
&= \sqrt{M}_N B_N + \sqrt{M}_N A_N.
\end{align*}


\section{The second piece of the numerator}

Looking at the second piece, recall that 
\begin{align*}
\sqrt{M_N} \frac{1}{\tilde{M}_N} \sum_{i=1}^{\tilde{M}_N} \left[ E[ \tilde{\omega}^{N,i} f(\tilde{\xi}^{N,i}) \mid \mathcal{F}^N] - \nu L(f) \right]
&= \sqrt{M_N}  M_N^{-1} \sum_{i=1}^{M_N}  \left[ L(\xi^{N,i},f) - \nu L(f)\right]\\ &\overset{D}{\to} N( , \sigma^2(Lf)) .
\end{align*}
The convergence takes place because $f$ is assumed to be from $\tilde{\mathsf{A}}$, and this implies that, by definition, $L(x,f) \in \mathsf{A}$. 

Note that this variance is taken with respect to the previous time's measure, and it is the variance of a conditional expectation. This is half of the law of total variance.

\section{The first piece of the numerator}

The second part of the numerator is 

$$
\sqrt{\tilde{M}_N}\left\{  \frac{1}{\tilde{M}_N} \sum_{i=1}^{\tilde{M}_N} \tilde{\omega}^{N,i} f(\tilde{\xi}^{N,i}) -  \frac{1}{\tilde{M}_N} \sum_{i=1}^{\tilde{M}_N} E[ \tilde{\omega}^{N,i} f(\tilde{\xi}^{N,i}) \mid \mathcal{F}^N] \right\} = \sqrt{\tilde{M}_N} B_N
$$

We're going to apply Proposition 9.5.12 to it. We make the substitution $V_{N,j} = \tilde{M}_N^{-1/2} \tilde{\omega}^{N,j} f(\tilde{\omega}^{N,j})$, and verify the assumptions one by one. 
\newline

\subsection{First Condition}


The triangular array is conditionally independent given $\mathcal{F}^N$, and for any row/column, $E[ \{\tilde{\omega}^{N,j}\}^2 \{f(\tilde{\omega}^{N,j})\}^2 \mid \mathcal{F}^N] < \infty$

This is true because we are only looking at functions $f$ in 
$$
\tilde{\mathsf{A}} \overset{\text{def}}{=} \left\{ f \in L^2(\mathsf{X}, \mu) : L(x,f) \in \mathsf{A}, x \mapsto \int R(x,dx')\left[ \frac{dL(x,\cdot)}{dR(x,\cdot)}(x') f(x') \right]^2 \in \mathsf{C} \right\}.
$$



\subsection{Second Condition}

There exists a constant $\sigma^2 >0$ such that
$$
\tilde{M}_N^{-1}\sum_{j=1}^{\tilde{M}_N} \text{Var}\left[  \tilde{\omega}^{N,j} f(\tilde{\omega}^{N,j})  \mid \mathcal{F}^N \right] \overset{\text{p}}{\to} \sigma^2
$$

This is easy to show if you split up the variance into the mean of the square minus the square of the mean:
$$
\tilde{M}_N^{-1}\sum_{j=1}^{\tilde{M}_N} E\left[  \{\tilde{\omega}^{N,j}\}^2 \{f(\tilde{\omega}^{N,j})\}^2  \mid \mathcal{F}^N \right] - \tilde{M}_N^{-1}\sum_{j=1}^{\tilde{M}_N} \{E\left[  \tilde{\omega}^{N,j} f(\tilde{\omega}^{N,j})  \mid \mathcal{F}^N \right]\}^2
$$

First, $f \in \tilde{\mathsf{A}}$, so 
$$
\tilde{M}_N^{-1}\sum_{j=1}^{\tilde{M}_N} E\left[  \{\tilde{\omega}^{N,j}\}^2 \{f(\tilde{\omega}^{N,j})\}^2  \mid \mathcal{F}^N \right] \overset{\text{p}}{\to}
\iint \nu(dx) R(x,dx')\left[ \frac{dL(x,\cdot)}{dR(x,\cdot)}(x') f(x') \right]^2
$$
it might be helpful to write that last piece as an expectation:
$$
\iint \nu(dx) R(x,dx')\left[ \frac{dL(x,\cdot)}{dR(x,\cdot)}(x') f(x') \right]^2
=
[\nu \otimes R]\left\{ \left[ \frac{dL(x,\cdot)}{dR(x,\cdot)}(x') f(x') \right]^2 \right\}
$$

The second part also converges too. This is because 
$$
x \mapsto \left[ \int R(x,dx') \frac{dL(x,\cdot)}{dR(x,\cdot)}(x') f(x') \right]^2 \in \mathsf{C}
$$

because

$$
\left[ \int R(x,dx') \frac{dL(x,\cdot)}{dR(x,\cdot)}(x') f(x') \right]^2 \le \int R(x,dx')\left[ \frac{dL(x,\cdot)}{dR(x,\cdot)}(x') f(x') \right]^2
$$
of Jensen's inequality. The right hand side is in $\mathsf{C}$, and because it is proper, the left hand side is too. This means that 

\begin{align*}
\tilde{M}_N^{-1}\sum_{j=1}^{\tilde{M}_N} \{E\left[  \tilde{\omega}^{N,j} f(\tilde{\omega}^{N,j})  \mid \mathcal{F}^N \right]\}^2 
&\overset{\text{p}}{\to} \nu\left[ \left[ \int R(x,dx') \frac{dL(x,\cdot)}{dR(x,\cdot)}(x') f(x') \right]^2  \right] \\
&= \nu\left(\left[L(x,f) \right]^2\right)
\end{align*}


\subsection{Third Condition}


The third condition of 9.5.12 is the Lindberg condition: for all $\epsilon > 0$,
$$
\tilde{M}_N^{-1} \sum_{j=1}^{\tilde{M}_N} [\tilde{\omega}^{N,j}]^2 [f(\tilde{\xi}^{N,j})]^2 \mathbf{1}\left\{ |\tilde{\omega}^{N,j} f(\tilde{\xi}^{N,j}) | \ge \epsilon\sqrt{\tilde{M}_N}   \right\}
$$



This is true, and we're going to use the dominated convergence argument again here. After picking an $\epsilon$, pick an arbitrary $C > 0$ and notice that

\begin{align*}
& \lim_{\tilde{M}_N \to \infty} \tilde{M}_N^{-1} \sum_{j=1}^{\tilde{M}_N} E[ \{\tilde{\omega}^{N,j}\}^2 \{ f(\tilde{\xi}^{N,j})\}^2 1\left( |\tilde{\omega}^{N,j} f(\tilde{\xi}^{N,j})| \ge \sqrt{\tilde{M}_N} \epsilon \right) \mid \mathcal{F}^N]  \\
&\le \lim_{\tilde{M}_N \to \infty} \tilde{M}_N^{-1} \sum_{j=1}^{\tilde{M}_N} E[  \{\tilde{\omega}^{N,j}\}^2 \{ f(\tilde{\xi}^{N,j})\}^2 1\left( |\tilde{\omega}^{N,j} f(\tilde{\xi}^{N,j})| \ge C \right) \mid \mathcal{F}^N]  \tag{if $C \le \sqrt{\tilde{M}_N} \epsilon$} \\
&= \nu\left\{ E[ \tilde{\omega}^{N,j} f(\tilde{\xi}^{N,j}) 1\left( |\tilde{\omega}^{N,j} f(\tilde{\xi}^{N,j})| \ge C \right) \mid \mathcal{F}^N] \right\}
\end{align*}

The last line is true because $E[  \{\tilde{\omega}^{N,j}\}^2 \{ f(\tilde{\xi}^{N,j})\}^2 1\left( |\tilde{\omega}^{N,j} f(\tilde{\xi}^{N,j})| \ge C \right) \mid \mathcal{F}^N]   \in \mathsf{C}$. Why? Well by assumption, $f \in \tilde{\mathsf{A}}$, and by definition of that set of functions, $E[  \{\tilde{\omega}^{N,j}\}^2 \{ f(\tilde{\xi}^{N,j})\}^2  \mid \mathcal{F}^N]   \in \mathsf{C}$. This one is larger than the other, so proprietary guarantees it works. 


Following up on that we have 
\begin{align*}
0 &\le \lim_{M_N \to \infty} \sum_{j=1}^{\tilde{M}_N} E[V_{N,j}^2 1\left( |V_{N,j}| \ge \epsilon \right) \mid \mathcal{F}^N] \\
&= \lim_{C \to \infty} \lim_{M_N \to \infty} \sum_{j=1}^{\tilde{M}_N} E[V_{N,j}^2 1\left( |V_{N,j}| \ge \epsilon \right) \mid \mathcal{F}^N] \tag{no $C$ so limit doesn't matter}\\
&\le \lim_{C \to \infty} \nu\left\{ E[ \tilde{\omega}^{N,j} f(\tilde{\xi}^{N,j}) 1\left( |\tilde{\omega}^{N,j} f(\tilde{\xi}^{N,j})| \ge C \right) \mid \mathcal{F}^N] \right\} \tag{above work} \\
&= \nu\left\{ \lim_{C \to \infty} E[ \tilde{\omega}^{N,j} f(\tilde{\xi}^{N,j}) 1\left( |\tilde{\omega}^{N,j} f(\tilde{\xi}^{N,j})| \ge C \right) \mid \mathcal{F}^N] \right\} \tag{DCT} \\
&= 0.
\end{align*}
Therefore condition iii of 9.5.12 is satisfied, and we have the desired result:


\begin{align*}
& \sqrt{\tilde{M}_N}\left\{  \frac{1}{\tilde{M}_N} \sum_{i=1}^{\tilde{M}_N} \tilde{\omega}^{N,i} f(\tilde{\xi}^{N,i}) -  \frac{1}{\tilde{M}_N} \sum_{i=1}^{\tilde{M}_N} E[ \tilde{\omega}^{N,i} f(\tilde{\xi}^{N,i}) \mid \mathcal{F}^N] \right\} \\
& \overset{\text{D}}{\to}
\text{Normal}\left( 0, 
\iint \nu(dx) R(x,dx')\left[ \frac{dL(x,\cdot)}{dR(x,\cdot)}(x') f(x') \right]^2
-\nu\left(\left[L(x,f) \right]^2\right)\right)
\end{align*}



\subsection{Putting it All Together}


So far we have shown that the following two are asymptotically normal:
\begin{enumerate}
\item $M_N^{1/2}A_n = \sqrt{M_N} \frac{1}{\tilde{M}_N} \sum_{i=1}^{\tilde{M}_N} \left[ E[ \tilde{\omega}^{N,i} f(\tilde{\xi}^{N,i}) \mid \mathcal{F}^N] - \nu L(f) \right]$
\item $\tilde{M}_N^{1/2} B_n = \sqrt{\tilde{M}_N}\left\{  \frac{1}{\tilde{M}_N} \sum_{i=1}^{\tilde{M}_N} \tilde{\omega}^{N,i} f(\tilde{\xi}^{N,i}) -  \frac{1}{\tilde{M}_N} \sum_{i=1}^{\tilde{M}_N} E[ \tilde{\omega}^{N,i} f(\tilde{\xi}^{N,i}) \mid \mathcal{F}^N] \right\} $.
\end{enumerate}

To show their asymptotic *joint* distribution, we have to use the law of total expectation:

\begin{align*}
\lim_{N \to \infty}E\left[ \exp\left[i \left(sM_N^{1/2}A_n + t\tilde{M}_N^{1/2} B_n  \right) \right] \right] 
&= \lim_{N \to \infty}E\left[ \exp\left[i sM_N^{1/2}A_n + it\tilde{M}_N^{1/2} B_n   \right] \right] \\
&= \lim_{N \to \infty}E\left[ \exp\left[i sM_N^{1/2}A_n\right]\exp\left[ it\tilde{M}_N^{1/2} B_n   \right] \right] \\
&= \lim_{N \to \infty}E\left[ \exp\left[i sM_N^{1/2}A_n\right] E\left( \exp\left[ it\tilde{M}_N^{1/2} B_n   \right] \mid \mathcal{F}^N\right) \right] \\
&= E\left[ \lim_{N \to \infty}\exp\left[i sM_N^{1/2}A_n\right]\lim_{N \to \infty} E\left( \exp\left[ it\tilde{M}_N^{1/2} B_n   \right] \mid \mathcal{F}^N\right) \right] \tag{DCT} \\
&= E\left[ \lim_{N \to \infty}\exp\left[i sM_N^{1/2}A_n\right] \exp\left(-.5t^2\eta^2(f) \right) \right] \\
&= \exp\left(-.5t^2\eta^2(f) \right) E\left[ \lim_{N \to \infty}\exp\left[i sM_N^{1/2}A_n\right]  \right] \\
&= \exp\left(-.5t^2\eta^2(f) \right) \lim_{N \to \infty} E\left[ \exp\left[i sM_N^{1/2}A_n\right]  \right] \tag{DCT} \\
&= \exp\left(-.5t^2\eta^2(f) \right) \exp\left(- .5s^2\sigma^2(Lf) \right)
\end{align*}

The second to last move should seem familiar:
$$
M_N^{1/2}(A_N + B_N) = M_N^{1/2}A_N + \frac{1}{\sqrt{\alpha_N}} \tilde{M}_N^{1/2} B_N \overset{\text{D}}{\to} \text{Normal}\left[0, \sigma^2(Lf) + \frac{1}{\alpha}\eta^2(f) \right].
$$

The final move is using Slutsky's:

$$
\frac{M_N^{1/2}(A_N + B_N)}{\frac{1}{\tilde{M}_N} \sum_{j=1}^{\tilde{M}_N} \tilde{\omega}^{N,j}}  \overset{\text{D}}{\to} \text{Normal}\left[0, \frac{\sigma^2(Lf) + \frac{1}{\alpha} \eta^2(f)}{[\nu L (f) ]^2} \right].
$$

\end{document}