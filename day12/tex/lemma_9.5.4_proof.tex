\documentclass{article}

\usepackage{amsmath}
\usepackage{amsfonts}
\usepackage{url}
\usepackage{bbm} %for bold indicator func

\title{A More Detailed Proof of Lemma 9.5.4}

\begin{document}
\maketitle

Assume
$$
|U_N| \le \eta V_N + W_N(\eta).
$$
Then, for any $\epsilon > 0$ and $\eta > 0$
\[
\{ |U_N| \le \epsilon \} \supset \{ V_N \le \epsilon/2\eta \}  \cap \{ W_N(\eta) \le \epsilon/2\}.
\]
Look at the contrapositive:
\[
\{ |U_N| > \epsilon \} \subset \{ V_N > \epsilon/2\eta \}  \cup \{ W_N(\eta) > \epsilon/2\}.
\]
That means
\begin{align*}
P(|U_N| > \epsilon) &\le P(\{ V_N > \epsilon/2\eta \}  \cup \{ W_N(\eta) > \epsilon/2\}) \\
&\le P( V_N > \epsilon/2\eta ) + P( W_N(\eta) > \epsilon/2) \tag{sub-additivity}
\end{align*}
which is the first equation of the authors' proof. 
\newline


Taking the $\limsup_{N \to \infty}$ on both sides we have
\begin{align*}
\limsup_N P(|U_N| > \epsilon) &\le \limsup_N \{ P( V_N > \epsilon/2\eta ) + P( W_N(\eta) > \epsilon/2) \}\\
&\le \limsup_N P(V_N >\epsilon/2\eta) + \limsup_N P( W_N(\eta) >\epsilon/2 ) \tag{properties of sup} \\
&= \limsup_N P(V_N > \epsilon/2\eta) + \lim_N P( W_N(\eta) > \epsilon/2 ) \tag{assumption}\\
&= \limsup_N P(V_N > \epsilon/2\eta) + 0 \tag{assumption}\\
&= 0.
\end{align*}


\end{document}